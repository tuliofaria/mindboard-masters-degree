\section{Requisitos funcionais}

\subsection{RF 01 – Identificar professores e alunos}

\textbf{Descrição:} a ferramenta deve permitir a identificação e autenticação de usuários professores e alunos, bem como gerenciar permissões específicas a cada um deles. Além disso, a forma de autenticação deverá exigir o preenchimento mínimo de informações.

\textbf{Prioridade:} essencial.

\subsection{RF 02 – Gerenciar aulas}

\textbf{Descrição:} a ferramenta deve permitir que professores iniciem novas aulas. Aulas são definidas como um conjunto de ações durante um período de tempo. Estão ações totalizam em conteúdo gerado.

Uma aula poderá ser identificada de forma única e simples, e poderá ser removida ou alterada pelos professores autores.

\textbf{Prioridade:} essencial.

\subsection{RF 03 – Acessar aulas}

\textbf{Descrição:} a ferramenta deve permitir que alunos e professores acessem aulas já iniciadas. Quando acessado como aluno, as ações realizadas são apenas visíveis ao próprio aluno, já quando acesso como professor, as ações realizadas serão distribuídas entre todos os participantes da aula.

Um único usuário poderá acessar uma mesma aula em mais de um dispositivo diferente. Permitindo, por exemplo, que professores acessem uma versão para projeção e outra gerenciar o conteúdo. Ou mesmo um aluno, acesse ao mesmo tempo de um \emph{tablet} e \emph{notebook}.

\textbf{Prioridade:} essencial.

\subsection{RF 04 – Projetar apresentações de slides em uma aula}

\textbf{Descrição:} a ferramenta deve permitir que professores insiram e projetem apresentações de \emph{slides} dentro de uma aula. Uma vez inserido e gerenciado a apresentação de \emph{slides} (por exemplo, alterar do \emph{slide} 1 para o \emph{slide} 2), todos os participantes receberão e visualizarão o mesmo \emph{slide}.

\textbf{Prioridade:} essencial.

\subsection{RF 05 – Anotações}

\textbf{Descrição:} a ferramenta deve permitir que alunos e professores anotem o conteúdo através de ferramentas de desenho. As anotações feitas por professores serão distribuídas a todos os alunos, já as realizadas pelos alunos serão disponíveis apenas para os que as criaram.

\textbf{Prioridade:} essencial.

\subsection{RF 06 – Quiz}

\textbf{Descrição:} a ferramenta deve permitir que professores realizem pequenas enquetes durante uma sessão. As respostas destas questões deverão serem totalizadas em tempo real ao professor, que pode escolher se quer exibir os resultados a todos os participantes.

\textbf{Prioridade:} desejado.

\subsection{RF 07 – Coletar e sumarizar opiniões sobre a aula}

\textbf{Descrição:} durante uma aula, os alunos poderão indicar seus relativos níveis de satisfação imediato ao professor. Este nível de satisfação é exibido em tempo real ao professor, e permite uma análise imediata do andamento de uma aula. Os níveis de satisfação coletados serão: positivo, neutro e negativo. Esta informação será armazenada também para posterior análise pelo professor.

\textbf{Prioridade:} essencial.

\subsection{RF 08 – Coletar e sumarizar dúvidas durante e após a aula}

\textbf{Descrição:} permitir que alunos façam perguntas durante e após as aula. No momento de realizar uma pergunta, o aluno pode informar se é apenas para o professor ou se pode ser respondida por um outro aluno. Ambos, professores e alunos, podem continuar colaborando sobre a dúvida postada.

A dúvida também pode ser contextualizada sendo indicado o momento em que ela ocorreu dentro da aula. Em sua resposta, podem ser citados momentos específicos da aula (atual ou outras aulas).

\textbf{Prioridade:} essencial.


\subsection{RF 09 – Rever uma aula}

\textbf{Descrição:} após o fim de uma aula, todos os usuários poderão rever o conteúdo construído, podendo controlar o fluxo através de uma linha do tempo.

\textbf{Prioridade:} essencial.

\subsection{RF 10 – Colaborar assincronamente em uma aula}

\textbf{Descrição:} após o fim de uma aula, todos os usuários poderão rever a mesma e colaborar adicionando anotações. Exibindo para o usuário uma linha do tempo com as ações que ocorreram durante a aula.

\textbf{Prioridade:} essencial.

\subsection{RF 11 – Permitir o compartilhamento de código-fonte em tempo-real}

\textbf{Descrição:} permitir que o professor construa códigos de linguagens de programação e que estes códigos sejam compartilhados em tempo-real com os alunos.

\textbf{Prioridade:} essencial.

\subsection{RF 12 - Permitir integração com outros sistemas}

\textbf{Descrição:} permitir que outros sistemas interajam programaticamente com o Mindboard e realize as seguintes operações:

- Identificar e autenticar usuários;

- Gerenciar aulas e permitir que as mesmas sejam acessadas diretamente do outro sistema;

\textbf{Prioridade:} desejado.

O requisito \textbf{RF 06} teve sua prioridade alterada de essencial para desejado para que houvesse tempo hábil de trocar o requisito \textbf{RF 11} para essencial. Esta troca foi necessária pois o compartilhamente de código-fonte é essencial ao curso de verão que foi realizado no experimento, e também foi possível, pois haveria outras formas de colaboração como as anotações (RF 10)  e também por esta funcionalidade estar disponíveis em SGC que podem ser integrados no Mindboard no futuro.

\section{Requisitos não funcionais}

Os requisitos não funcionais incluem os requisitos de desempenho e outros atributos de qualidade do produto, como usabilidade, padrões de codificação e implementação \cite{padua05}. Os requisitos não funcionais esperados para a ferramenta Mindboard são descritos a seguir.

\textbf{RNF 1} – A ferramenta Mindboard deve possuir uma arquitetura facilmente extensível, permitindo futuras funcionalidades de serem acrescentadas sem que seja necessário remodelar toda a ferramenta.

Este requisito não funcional será atingido definindo uma arquitetura em módulos e através do padrão de projetos \emph{Model}, \emph{View} e \emph{Controller} (MVC). Com esta organização do código será possível adicionar novas funcionalidades ao sistema sem a necessidade de refazer partes de códigos já funcionais.


\textbf{RNF 2} – A ferramenta Mindboard deve possuir suporte a computadores e dispositivos móveis, possuindo uma interface gráfica adaptável a plataforma utilizada.

Este requisito não funcional será obtido através do desenvolvimento de interfaces responsivas utilizando HTML, CSS e JavaScript.

\textbf{RNF 3} – A ferramenta Mindboard deve possuir uma forma de acesso simplificada, reduzindo assim barreiras de utilização. 

Para atender este requisito serão permitidas duas formas de acesso. A primeira através de um endereço curto, e uma segunda maneira, através da representação gráfica em um QR-Code deste endereço curto. Assim, será possível a digitação rápida do endereço ou a leitura direta por dispositivos móveis (no caso do QR-Code).

\textbf{RNF 4} – A ferramenta Mindboard deve permitir que um mesmo usuário acesse de dispositivos diferentes dentro de uma mesma sessão. 

A ferramenta deverá, para atender este requisito, não distinguir e não restringir o número de acessos por um mesmo usuário.

\textbf{RNF 5} – A ferramenta Mindboard deve ser escalável, permitindo atender um grande número de usuários simultâneos apenas aumentando os recursos da infraestrutura, como por exemplo, aumentando o número de servidores, ou ampliando a quantidade de memória e processamento.  Para permitir a escalabilidade, a arquitetura da ferramenta será definida utilizando balanceadores de carga, servidores redundantes de aplicação, servidores redundantes de bancos de dados e servidores replicados de banco de dados em memória. A arquitetura escalável é mostrada na Seção \ref{sec:tecnologias}.